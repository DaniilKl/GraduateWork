\documentclass[a4paper, 10pt, twoside]{article}

%Preamble:
\usepackage{Preamble}

%Main part of the document:
\begin{document}
%Title page from PDF:
\includepdf[pages=-]{PDFs/StronaTytulowa_PracaDyplomowa}%The tittle is downloaded from university page

%Page counter start:
\setcounter{page}{1}

%Abstract:
    %Abstract using the Polish language:
    \section*{Streszczenie}\addcontentsline{toc}{section}{Streszczenie}
    \subfile{Sections/AbstractPL}

    %Abstract using the English language:
    \section*{Abstract}\addcontentsline{toc}{section}{Abstract}
    \subfile{Sections/AbstractEN}

%End of the abstract.

%Table of content:
\addcontentsline{toc}{section}{Spis treści}
\tableofcontents
%End of the Table of content.

% List of figures:
\listoffigures

%List of symbols and abbreviations:
% TODO: ekcja "Skróty i oznaczenia" nie jest dodawana automatycznie do spisu treści, trzeba ją dodawać ręczenie. Ale przy ręcznym dodawaniu, spis treści nie wskazuje prawidłowo na początek sekcji, tylko na początek strony gzdie ta sekcja się znajduje.
\addcontentsline{toc}{section}{Skróty i oznaczenia}
\printglossary[type=\acronymtype,nonumberlist]

%Main sections (numerated):

%Intro:
\section{Wstęp}
\subfile{Sections/MainSections/00-Intro}

\section{Architektury i przykłady realizazcji}
\subfile{Sections/MainSections/01-basics}

\section{Dekompozycja i wyodrębnienie}
\subfile{Sections/MainSections/02-decomposition-and-separation}

\section{Analiza}
\subfile{Sections/MainSections/03-analysis}

%End of main sections.

%The summary:
\addcontentsline{toc}{section}{Podsumowanie}
\section*{Podsumowanie}

\subfile{Sections/Summary}
\acrshort{rtos}
%End of the Summary.

%The bibliography section:
\addcontentsline{toc}{section}{Wykaz literatury}
\printbibliography[title={Wykaz literatury}]
%End of the bibliography section.

%Appendices:
\subfile{Sections/Appendices}

%The End.
\end{document}
