\documentclass[../main]{subfiles}
\graphicspath{{\subfix{../Images/}}}

\begin{document}

\section*{Załącznik nr 1: System wbudowany}\addcontentsline{toc}{section}{Załącznik nr 1: System wbudowany}\label{sec:zalacznik-1}

Skoro nie ma jednoznacznej definicji pojęcia "system wbudowany" - należy określić definicję, która jest
wykorzystywana w tej pracę.

System wbudowany jest systemem, to znaczy ma cechy systemu: jest złożony z więcej niż jednego elementu,
jest funkcjonalnie niezależny od środowiska (tzn. może być wyodrębniony ze środowiska) i posiada
możliwość reagowania (tzn. odbierania, przetwarzania i odpowiedzi) % TODO: odbierania? możliwość?
na bodźce zewnętrzne. "wbudowany" oznacza, że system jest częścią innego systemu fizycznie i/lub
funkcjonalnie.

Natomiast przełożyć tą definicję na dziedzinę elektroniki i informatyki można w następujący sposób:
system wbudowany — jest to system komputerowy składający się z jednostki obliczeniowej i modułów
wejścia/wyjścia (tzn. może reagować na bodźce zewnętrzne), mający wszystkie narzędzia (w tym
oprogramowanie) dla spełnienia pewnej, zdefiniowanej funkcji (więc może być wyodrębniony) i będący
częścią większego systemu.
% TODO: Nie cytuję tu rzadnej książki ani normy, bazuję sie na swojej wiedzę, ale agólnie ta definicja jest podobna od książki do książki.

\section*{Załącznik nr 2: ARM TrustZone}\addcontentsline{toc}{section}{Załącznik nr 2: ARM TrustZone}\label{sec:zalacznik-2}
\begin{figure}
    \centering
    \includegraphics[width=0.95\textwidth]{Images/trustzone-m.png}
    \caption{TrustZone dla architektury \acrshort{arm} v8-M}
    \label{fig:trustzone-m}
\end{figure}
\begin{figure}
    \centering
    \includegraphics[width=0.95\textwidth]{Images/trustzone-a.png}
    \caption{TrustZone dla architektury \acrshort{arm} v8-A}
    \label{fig:trustzone-a}
\end{figure}

Koncept technologi TrustZone bazuję się na podziale sprzętowym środowiska (pamięci i innych zasobów) na
środowisko zaufane i niezaufane (\cref{fig:trustzone-m} i
\cref{fig:trustzone-a}), co jest robione za pomocą rozszerzeń do architektur ARMv8-A i
ARMv8-M dotyczących \acrshort{bsa} i \acrshort{isa}.

Dla architektury ARMv8-A został dodany \acrshort{el}3, w którym jest zamieszczony monitor bezpieczeństwa
(\cref{fig:trustzone-a}), realizowany przez oprogramowanie \acrshort{arm}'owe Trusted
Firmware A. Faktycznie ten monitor bezpieczeństwa jest pewnym mostem przekazującym polecenia ze
środowiska niezaufanego do środowiska zaufanego oraz przełączający procesor w bezpieczny tryb za pomocą
\acrshort{smccc}. \cite{smccc}\cite{trustzoneaarch64} % TODO: automatic images indexing

Dla architektury ARMv8-M zostały dodane rozszerzenia pozwalające na separację środowiska zaufanego od
niezaufanego (\cref{fig:trustzone-m}) za pomocą modułów \acrshort{sau} i \acrshort{idau},
oraz instrukcji dla komunikacji pomiędzy środowiskami: \acrshort{isa} SG, BXNS i BLXNS.
\cite{trustzonearmv8m}

\end{document}
