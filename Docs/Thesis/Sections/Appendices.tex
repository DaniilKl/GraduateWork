\documentclass[../main]{subfiles}
\graphicspath{{\subfix{../Images/}}}

\begin{document}

\section*{Załącznik nr 1: System wbudowany}\addcontentsline{toc}{section}{Załącznik nr 1: System wbudowany}\label{sec:zalacznik-1}

Skoro nie ma jednoznacznej definicji pojęcia "system wbudowany" - należy określić definicję, która jest
wykorzystywana w tej pracę.

System wbudowany jest systemem, to znaczy ma cechy systemu: jest złożony z więcej niż jednego elementu,
jest funkcjonalnie niezależny od środowiska (tzn. może być wyodrębniony ze środowiska) i posiada
możliwość reagowania (tzn. odbierania, przetwarzania i odpowiedzi) % TODO: odbierania? możliwość?
na bodźce zewnętrzne. "wbudowany" oznacza, że system jest częścią innego systemu fizycznie i/lub
funkcjonalnie.

Natomiast przełożyć tą definicję na dziedzinę elektroniki i informatyki można w następujący sposób:
system wbudowany — jest to system komputerowy składający się z jednostki obliczeniowej i modułów
wejścia/wyjścia (tzn. może reagować na bodźce zewnętrzne), mający wszystkie narzędzia (w tym
oprogramowanie) dla spełnienia pewnej, zdefiniowanej funkcji (więc może być wyodrębniony) i będący
częścią większego systemu.
% TODO: Nie cytuję tu rzadnej książki ani normy, bazuję sie na swojej wiedzę, ale agólnie ta definicja jest podobna od książki do książki.

\end{document}