\documentclass[../main]{subfiles}
\graphicspath{{\subfix{../Images/}}}

\begin{document}
\paragraph{}
W moment napisania \todo{w moment napisania?} pracy systemy cyfrowe i oprogramowanie zarządzające tymi
systemami są bardzo rozbudowane posiadające od kilku do kilkunastu poziomów abstrakcji i podsystemów. Z
powodu poważnego obniżenia ceny na technologię półprzewodnikową i duży skok progresu problemy małej
pamięci i małej wydajności systemów cyfrowych są przeszłością. Z kolei stare problemy zostały
zastąpione innymi, wynikającymi z coraz większej skomplikowaności systemów i wzrostem zapotrzebowań:
problem bezpieczeństwa i problem skalowalności. Aktualnie dużo zasobów jest spędzane na dobudowanie
kolejnego poziomu na już istniejący system i rzadko jest zwracana uwaga, co leży pod spodem, chociaż
jest tam logika nie mniej interesująca i definiująca właściwości całego systemu.

\paragraph{}
Centrum uwagi \todo{centrum uwagi?} tej pracy jest logika zarządzająca zasobami obliczeniowymi systemu
cyfrowego, nazywana ogólnie systemem operacyjnym, a szczególnie oprogramowanie planujące wykonywanie zadań
i zarządzające zasobami jednostki obliczeniowej systemu — procesora. Praca zacznie się z podstawowych
pojęć i architektur systemowych, i będzie coraz więcej pogłębiać się, aż nie dojdzie do pierwotnie
postawionego celu. Następnie wspomniane przed tym programy zostaną przeanalizowane ze strony
konceptualne (pomysł, idea), formalnej (specyfikacje, definicje, założenia) i praktycznej (realizacje,
implementacje). Zostaną też przeprowadzone eksperymenty praktyczne w celu pogłębienia i systematyzacji
wiedzy. Na końcu autor przedstawi własne przemyślenia i wnioski bazujące się na poprzedniej pracę
teoretycznej i praktycznej.
\\
\\
\textbf{Słowa kluczowe:} systemy operacyjne, systemy cyfrowe, zarządzanie zasobami procesora
\todo{słowa kluczowe}
\\
\textbf{Dziedzina nauki i techniki, zgodnie z wymaganiami OECD:} smth \todo{dziedzina nauki OECD}

\end{document}
