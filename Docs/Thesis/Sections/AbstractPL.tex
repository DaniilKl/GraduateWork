\documentclass[../main]{subfiles}
\graphicspath{{\subfix{../Images/}}}

\begin{document}

W momencie napisania pracy systemy cyfrowe i oprogramowanie zarządzające tymi systemami są bardzo rozbudowane, posiadające od kilku do kilkunastu poziomów abstrakcji i podsystemów. Z powodu znacznego obniżenia ceny technologii półprzewodnikowej i dużego skoku technologicznego, problemy małej pamięci i niskiej wydajności systemów cyfrowych są już przeszłością. Aktualnie dużo zasobów jest poświęcane na dodawanie kolejnego poziomu do już istniejących systemów, a rzadko jest zwracana uwaga na to, co leży pod spodem, chociaż tam znajduje się logika nie mniej interesująca i definiująca właściwości całego systemu.

Centrum uwagi tej pracy stanowi logika zarządzająca zasobami obliczeniowymi systemu cyfrowego, nazywana ogólnie systemem operacyjnym, a w szczególności oprogramowanie planujące wykonywanie zadań i zarządzające zasobami jednostki obliczeniowej systemu — procesora.

Praca rozpoczyna się od podstawowych pojęć i architektur systemowych, a następnie stopniowo pogłębia omawiane zagadnienia, aż do programów zarządzających zasobami jednostki obliczeniowej. Następnie omówione wcześniej programy zostaną przeanalizowane. W ostatniej części pracy kilka algorytmów zarządzania jednostką obliczeniową zostaną zaimplementowane w realnym systemie operacyjnym w celu zbadania ich cech. Na końcu autor przedstawi własne przemyślenia i wnioski, bazując na poprzedniej pracy teoretycznej i praktycznej.

\noindent\textbf{Słowa kluczowe:} system operacyjny, hiperwizor, system cyfrowy, zarządzanie zasobami procesora, planowanie zadań, FreeRTOS.
% TODO: słowa kluczowe

\noindent\textbf{Dziedzina nauki i techniki, zgodnie z wymaganiami OECD:} 2.2 Elektrotechnika, elektronika, inżynieria informatyczna; 2.2.b Robotyka i automatyka

\end{document}