\documentclass[../main]{subfiles}
\graphicspath{{\subfix{../Images/}}}

\begin{document}

At the time of writing, the digital systems and software running on these systems are very complex, having from several to dozen layers of abstractions and subsystems. Due to the significant reduction in the price of semiconductor technology and a large leap in progress, the problems of small memory and low efficiency of digital systems are a thing of the past. Currently, a lot of resources are spent on adding another abstraction level to existing systems, paying rather low attention to what is hidden underneath, although there is logic there that is no less interesting and defines the properties of the entire system.

The focus of this work is the logic managing the computational resources of the digital system, generally called the operating system, and, in particular, the software scheduling the execution of tasks and managing the resources of the system's computing unit - the processor.

The work will start with basic concepts and system architectures, and will delve deeper and deeper until it reaches the original goal. Then the programs mentioned before will be analyzed. In the last part of the work, several algorithms for managing computational unit resources will be implemented in a real operating system in order to investigate their characteristics. At the end, the author will present his own thoughts and conclusions based on the previous theoretical and practical work.

\textbf{Keywords:} operating system, hypervisor, digital system, scheduling, FreeRTOS.

\end{document}