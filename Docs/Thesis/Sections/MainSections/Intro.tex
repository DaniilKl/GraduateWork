\documentclass[../main]{subfiles}
\graphicspath{{\subfix{../../Images/}}}

\begin{document}

\subsection{Wprowadzenie}

\paragraph{}
Od momentu pojawienia się \todo{pojaiwienia się?} pierwszych urządzeń elektronicznych i maszyn
obliczeniowych minęło dość dużo czasu, w ciągu którego progres pchał \todo{progres pchał?} cele
postawione przed systemami cyfrowymi, co powodowało ich szybką ewolucję. Od listy instrukcji napisanej
na kartkach perforowanych, do skomplikowanych programów zajmujących duże obszary pamięci, dodaniu do
mocy obliczeniowej mnóstwa urządzeń peryferyjnych i prób abstrahować się od ich implementacji,
wprowadzenia skomplikowanych abstrakcji systemowych dla ułatwienia developmętu \todo{developmętu?} i
bezpieczeństwa, i, w końcu, łączenia mnóstwa systemów w jeden, mnogofunkcjonalny system.
\todo{mnogofunkcjonalny?}\todo{check punktuation}

\paragraph{}
Aktualnie, logika zarządzająca systemami cyfrowymi i sprawiająca, że system wykonuje postawione przed
nim cele oraz spełnia określone oczekiwania, jest bardzo skomplikowana, posiada dużo poziomów
abstrakcji i podsystemów, co znacząco utrudnia integrację i analizę. Dodatkowo, nieznajomość działania
\todo{Dodatkowo, nieznajomość?} tych systemów coraz częściej powoduje błędy i problemy podczas
implementacji rozwiązań lub realizacji celów bazujących się na tych systemach.

\paragraph{}
Z kolei, bardzo ważną dla automatyki częścią tego progresu jest też ewolucja systemów czasu
rzeczywistego. Jeszcze można spotkać się z systemami cyfrowymi pełniącymi rolę systemu czasu
rzeczywistego z oprogramowaniem wykonującym się wprost na sprzęcie (tzn. bez żadnych warstw abstrakcji
pomiędzy sprzętem a aplikacją, np. bez systemu operacyjnego). Jednak coraz częściej są spotykane
\acrshort{rtos}'y lub inne, bardziej skomplikowane oprogramowanie, które używa wzrost funkcjonalności i
wydajności \todo{używa wzrost?} mikrokontrolerów, na przykład coraz bardziej złożona wirtualizacja.
Wzbudzają zainteresowanie też sposoby i metody na realizację wymagań systemu czasu rzeczywistego, jak
również i sprawdzenie tych wymagań przy coraz bardziej skomplikowanym oprogramowaniu.

\subsection{Cel pracy}

\subsection{Zakres pracy}

\end{document}
