\documentclass[../../main]{subfiles}
\graphicspath{{\subfix{../../Images/}}}

\begin{document}

\subsection{Podstawowe architektury}

\todo[inline]{Przed napisaniem tej części zastanawiałem się, jak podzielić miedzy sobą pojęcia
architektury, struktury i pojęcia z języka angielskiego - desing. Dlaczego? No bo conajmniej pierwsze
dwa pojęcia będą często używane w tej prace, i ja się obawiałem, że będę wykorzystywał je w własny,
niezdefiniowany sposób (bo nie znalazłem tak naprawdę dokładnej definicji, są tylko dyfinicje dla
poszczególnych branży, w których te pojęcia są używane).\\
Po przemyśleniu zdefiniowałem ję w następujący sposób:\\
Struktura - łączenie czegokolwiek posiadającego własciwości lub funkcjanolności jednego typu z
czymkolwiek posiadającym właściwości lub funkcjanolności takiego samego typu w określony
sposób dla stworzenia czegokolwiek z innymi właściwościami lub funkcjanolnościami.\\
Architektura - łączenie czegokolwiek posiadającego własciwości lub funkcjanolności jednego typu
z czymkolwiek posiadającym właściwości lub funkcjanolnościinnego typu w określony sposób dla
stworzenia czegokolwiek z innymi właściwościami lub funkcjanolnościami.\\
Design - struktura lub architektura stowrzona dla osiągniecia pewnego celu (IMHO najbliższe tłumaczenie na język Polski - projekt, chociaż w języku angielskim project =/= design).
Zostawiam to tu, aby nie zgubić. W pracę będę korzystał z tych pojęć zgodnie z tymi definicjami.}

Ten rozdział przedstawia i opisuje podstawowe architektury oprogramowania w systemach wbudowanych
bazujących się na jednostkach obliczeniowych skonstruowanych na podstawie architektury
\acrshort{arm}'owej.

Definiowanie pojęcia systemu wbudowanego nie jest zakresem tej pracy, jednak w
\hyperref[sec:zalacznik-1]{załączniku nr 1} jest określono, z jakiej
definicji korzystano podczas napisania.

\end{document}